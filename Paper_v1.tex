\documentclass[12pt]{article}
% Load packages
\usepackage{url}  % Formatting web addresses  
\usepackage{ifthen}  % Conditional 
\usepackage{multicol}   %Columns
\usepackage[utf8]{inputenc} %unicode support
\usepackage{amsmath}
\usepackage{amssymb}
\usepackage{epsfig}
\usepackage{epstopdf}
\usepackage{graphicx}
\usepackage[margin=0.1pt,font=footnotesize,labelfont=bf]{caption}
\usepackage{setspace}
%\usepackage{longtable}
\usepackage{colortbl}
%\usepackage{palatino,lettrine}
%\usepackage{times}
%\usepackage[applemac]{inputenc} %applemac support if unicode package fails
%\usepackage[latin1]{inputenc} %UNIX support if unicode package fails
\usepackage[wide]{sidecap}
%\usepackage[authoryear,round,comma,sort&compress]{natbib}
\usepackage[square,sort,comma,numbers]{natbib}
%\usepackage[authoryear,round]{natbib}
\usepackage{supertabular}
\usepackage{simplemargins}
\usepackage{comment}
\usepackage{lineno}

\urlstyle{rm}

%\textwidth = 6.50 in
%\textheight = 9.5 in
%\oddsidemargin =  0.0 in
%\evensidemargin = 0.0 in
%\topmargin = -0.50 in
%\headheight = 0.0 in
%\headsep = 0.25 in
%\parskip = 0.15in
%\linespread{1.75}
\doublespace

%\usepackage{geometry}
\usepackage{fullpage}

%\bibliographystyle{plain}
\bibliographystyle{plos2009}

\makeatletter
\renewcommand\subsection{\@startsection
	{subsection}{2}{0mm}
	{-0.05in}
	{-0.5\baselineskip}
	{\normalfont\normalsize\bfseries}}
\renewcommand\subsubsection{\@startsection
	{subsubsection}{2}{0mm}
	{-0.05in}
	{-0.5\baselineskip}
	{\normalfont\normalsize\itshape}}
\renewcommand\section{\@startsection
	{subsection}{2}{0mm}
	{-0.2in}
	{0.05\baselineskip}
	{\normalfont\large\bfseries}}	
\renewcommand\paragraph{\@startsection
	{paragraph}{2}{0mm}
	{-0.05in}
	{-0.5\baselineskip}
	{\normalfont\normalsize\itshape}}
\makeatother

%Review style settings
%\newenvironment{bmcformat}{\begin{raggedright}\baselineskip20pt\sloppy\setboolean{publ}{false}}{\end{raggedright}\baselineskip20pt\sloppy}

%Publication style settings

% Single space'd bib -
\setlength\bibsep{0pt}

\renewcommand{\rmdefault}{phv}\renewcommand{\sfdefault}{phv}
\newcommand{\norm}[1]{\left\lVert#1\right\rVert}

% Change the number format in the ref list -
\renewcommand{\bibnumfmt}[1]{#1.}

% Change Figure to Fig.
\renewcommand{\figurename}{Fig.}

% Begin ...
\begin{document}
\begin{titlepage}
{\par\centering\textbf{\Large An Effective Model of the Core Retinoic Acid induced HL-60 Differentiation Circuit}}
\vspace{0.05in}
{\par \centering \large{Ryan Tasseff$^{1}$, 
Holly A. Jensen$^{1}$, 
Johanna Congleton$^{2}$, Andrew Yen$^{2}$ and Jeffrey D. Varner$^{1,*}$}}
\vspace{0.10in}
{\par \centering \large{$^{1}$~School of Chemical and Biomolecular Engineering}}
{\par \centering \large{Cornell University, Ithaca NY 14853}}
\vspace{0.1in}
{\par \centering \large{$^{2}$~Department of Biomedical Sciences}}
{\par \centering \large{Cornell University, Ithaca NY 14853}}
\vspace{0.1in}
{\par \centering \textbf{Running Title:}~Effective HL60 differentiation circuit}
\vspace{0.1in}
{\par \centering \textbf{To be submitted:}~\emph{Processes}}
\vspace{0.5in}
{\par \centering $^{*}$Corresponding author:}
{\par \centering Jeffrey D. Varner,}
{\par \centering Associate Professor, School of Chemical and Biomolecular Engineering,}
{\par \centering 244 Olin Hall, Cornell University, Ithaca NY, 14853} 
{\par \centering Email: jdv27@cornell.edu} 
{\par \centering Phone: (607) 255 - 4258}
{\par \centering Fax: (607) 255 - 9166}
\end{titlepage}
\date{}
\thispagestyle{empty}
\pagebreak
%%%%%%%%%%%%%%%%%%%%%%%%%%%%%%%%%%%%%%%%%%%%%%%%%%%%%%%%%%%%%%%%%%%%%%%%%%%%%%%%%%%%%%%%%%%%%%%%%%%%%%%%%%%
%%%%%%%%%%%%%%%%%%%%%%%%%%%%%%%%%%%%%%%%%%%%%%%%%%%%%%%%%%%%%%%%%%%%%%%%%%%%%%%%%%%%%%%%%%%%%%%%%%%%%%%%%%%
\section*{Abstract}
The complexity of human differentiation networks makes analysis of their architecture and function challenging.  
Analysis of model systems, such as the lineage-uncommitted human myeloblastic leukemia cell line HL-60, 
could inform the analysis of more complex programs. 
HL-60 cells undergo myeloid differentiation along with G1/G0-arrest when exposed to All-Trans Retinoic Acid (ATRA). 
One of the defining features of this program is slowly induced, persistent MAPK activation.
The molecular mechanisms of ATRA-induced commitment, arrest and functional differentiation are only partially understood.
In this study, we explored the ATRA-inducible c-Raf, also known as Raf1 (Raf), 
interactome to determine the functional and regulatory architecture responsible for persistent MAPK activation in HL-60 cells. 
To better understand the role of Raf in the ATRA-induced differentiation program, 
we constructed an ensemble of mechanistic, mathematical models of the transcriptional and post-translational events driving ATRA-induced MAPK activation. 
Bifurcation analysis of this model predicted bistability in ppERK levels as a function of ATRA forcing. 
A functional consequence of this was the ability to lock the MAPK cascade into a self-sustaining activated state, even after ATRA removal.
These simulations were then qualitatively validated with ATRA washout experiments. 
Taken together, this study provided further details on sustained MAPK activation, mechanistic insight into cellular memory,
and proof-of-concept that a
combination of experimental and computational methods is an effective strategy for dissecting complex intracellular signaling programs.


{\noindent \textbf{Keywords:}~Cell free metabolism, Mathematical modeling}

\pagebreak

\setcounter{page}{1}

\linenumbers

\section*{Introduction}

Understanding differentiation, 
the process by which precursor cells become more specialized cell types, 
is an important challenge facing biology. 
If differentiation programs could be rationally manipulated, 
advanced therapies could be developed to treat a spectrum of cancers, spinal cord injuries and neurodegenerative disorders. 
However, to rationally reprogram differentiation networks, 
we must first understand their connectivity and regulation \cite{Young2011}. 
Lessons learned in model systems, such as the lineage-uncommitted human myeloblastic cell line HL-60,
could inform our analysis of more complex programs. 
HL-60 has been a durable experimental model since the 1970's to study differentiation \cite{Breitman1980}.
Depending upon the stimulus, HL-60 cells undergo G1/G0-arrest and myeloid or monocytic differentiation. 
All-Trans Retinoic Acid (ATRA) induces G1/G0-arrest and myeloid differentiation in HL-60 cells, 
whereas 1,25-dihydroxy vitamin D3 induces arrest with monocytic differentiation.
Commitment to cell cycle arrest and terminal differentiation requires approximately 48 hr of treatment,
during which HL-60 cells undergo approximately two division cycles.  
Interestingly, cells treated with ATRA for time periods shorter than the commitment phase retain a limited inheritable memory, 
which reduces the time required to reach commitment during subsequent ATRA exposure \cite{YEN1984}.

Sustained Mitogen-Activated Protein Kinase (MAPK) cascade
activation is a defining feature of ATRA-induced differentiation in HL-60 cells. 
ATRA drives slow yet sustained MEK-dependent activation of the Raf/MEK/ERK pathway, 
leading to arrest and functional differentiation \cite{Yen1998}.
MEK inhibition results in the loss of both ERK and Raf phosphorylation, 
as well as the failure to arrest and terminally differentiate \cite{Yen1998,Hong2001}.
At the transcriptional level, 
ATRA (and its metabolic products) are ligands for the hormone activated nuclear transcription factors Retinoic Acid Receptor (RAR) and Retinoid X Receptor (RXR) \cite{Mangelsdorf1990}.
Activation of both RAR and RXR is necessary for ATRA-induced Raf phosphorylation and MAPK activation,
suggesting that the initiation of MAPK signaling is partially transcriptionally regulated \cite{Hong2001}.
ATRA, through activation of a transcription factor complex including RAR and RXR, induces the expression of many proteins.
BLR1 also known as CXCR5, is a putative heterotrimeric Gq protein-coupled receptor 
that is necessary for MAPK activation, growth arrest and functional differentiation \cite{YEN1990,LIPP1994,WANG2004}.
BLR1 was identified as an early ATRA (or D3)-inducible gene in HL-60 cells using differential display \cite{YEN1990}. 
Studies of the BLR1 promoter identified a 5' 17bp GT box approximately 1 kb upstream of the transcriptional start that conferred ATRA responsiveness \cite{WANG2004}.
Additionally, members of the BLR1 transcriptional activator complex, e.g. NFATc3 and CREB, 
can be phosphorylated by ERK, JNK or p38 MAPK family members \cite{Yang2002}.
This suggests positive feedback between BLR1 expression and MAPK activation. 
BLR1 overexpression enhanced Raf phosphorylation and accelerated terminal differentiation, 
while BLR1 knock-out HL-60 cells failed to activate Raf or differentiate in the presence of ATRA \cite{Wang2008}.
Interestingly, both the knockdown or inhibition of Raf, also reduced BLR1 expression and functional differentiation \cite{Wang2008}.
A recent computational study of ATRA-induced differentiation in HL-60 cells suggested that the BLR1-MAPK positive feedback circuit was sufficient to explain ATRA-induced sustained MAPK activation and the expression of differentiation markers \cite{Tasseff2011}. 
Model analysis also suggested that Raf was the most distinct of the MAPK proteins.

A critical question is what other components of the MAPK positive feedback circuit are required to drive ATRA-induced functional differentiation of HL-60 cells. 
Wang and Yen showed that ectopic expression of the constitutively active CR3 domain of Raf1 restored ATRA-induced G0 arrest and differentiation in BLR1 knock-out cells \cite{Wang2008}. 
However, ectopic expression of Raf1 CR3 alone, in the absence of ATRA, failed to induce arrest or differentiation.
Thus, additional ATRA-inducible components must exist, 
which independently promote arrest and differentiation in the absence of BLR1.
In this study, we explored this hypothesis using a combination of experimental and computational tools. 
First, we explored the ATRA-inducible Raf interactome 
by surveying a panel of 19 possible binding partners using immunoprecipitation (IP), 
with and without ATRA and the Raf inhibitor GW5074.  
Initially, we expected increased ATRA-dependent association between Raf and kinases linked to BLR1 activity; 
however, this was not supported by data.
Instead, we found that the interaction between the guanine nucleotide exchange factor Vav1 and Raf was both ATRA-inducible and simultaneously sensitive to Raf inhibition. 
Next, we considered how MAPK activation and differentiation were affected by the inhibition of Raf kinase activity in the presence and absence of ATRA.
We showed that Raf activity was directly proportional to ERK phosphorylation and to functional differentiation processes such as the generation of reactive oxygen species (ROS). 
Moreover, interactions between Raf and kinase partners such as Akt or CK2, or the scaffolding protein 14-3-3 were largely insensitive to ATRA treatment. 
These studies established the working hypothesis that Vav1 (or potentially other ATRA-inducible proteins) acted as limiting members of a constitutively assembled trigger complex that propelled sustained MAPK activation, arrest and differentiation.   
We tested this hypothesis by constructing a mechanistic mathematical model of the Raf-Vav1 circuit, 
based on the IP and Western blot data presented in this study, 
and from previous literature. 
The proposed model architecture was consistent with the ATRA-induced sustained MAPK activation observed experimentally. 
Additionally, we found the Raf-Vav1 circuit possessed interesting dynamic features such as bistability, 
that could explain ATRA-induction memory effects.

\clearpage

\section*{Results}



\clearpage

\section*{Discussion}
In this study, we explored the functional and regulatory architecture responsible for ATRA-induced MAPK activation in HL-60 cells.
Retinoic acid is an embryonic morphogen, a cancer chemotherapeutic agent, as well as potent regulator of cell differentiation and proliferation. 
Using a human lineage-uncommitted myelo-monocytic precursor cell, 
it has been shown that ATRA-induced phenotypic conversion and G0 arrest depend on eliciting a hyperactive prolonged signal along the Raf/MEK/ERK axis \cite{Yen1998}. 
In particular, signaling involves two positive loops, which incorporates the BLR1 receptor and Raf kinase, putatively activated by S621 phosphorylation, 
that is seminal to sustaining the signal needed for differentiation \cite{Wang2008}. 
In genetically engineered BLR1 KO cells crippled for signaling and consequentially ability to differentiate, ectopic expression of Raf activity restores the cell's ability to differentiate. 
While the canonical paradigm of MAPK signaling is a transient signal that is mitogenic, a prolonged signal elicits cell differentiation and G0 arrest. 
Interest thus becomes focused on whether there is a Raf partner that is responsible for the prolonged signal in ATRA-treated cells. 
Surveying known MAPK regulatory molecules using Western blotting and IP we found that ATRA increased the expression of Vav1 as well as its immunoprecipitation with Raf phosphorylated at S621. 
A parsimonious effective mathematical model showed that a Raf-Siganlsome-mediated positive feedback loop accounted for sustained MAPK signaling and aspects of phenotypic conversion and G0 arrest. 
The significant finding was thus that ATRA-regulated Vav1 expression and partnering with Raf could drive sustained signaling and phenotypic shift. The model also predicted the observed hysteresis effect of ATRA where the cell retains a memory of transient ATRA exposure, betrayed by activated signaling and by accelerated differentiation upon re-exposure to ATRA.

\clearpage

\section*{Materials and Methods}

\subsection*{Cell Culture and Treatment.}
Human myeloblastic leukemia cells (HL-60 cells) were grown in a humidified atmosphere of 5\% CO$_2$ at 37$^{o}$C and maintained in RPMI 1640 from Gibco (Carlsbad, CA) 
supplemented with 5\% fetal bovine serum from Hyclone (Logan, UT) and 1x antibiotic/antimicotic (Sigma, St. Louis, MO).
Cells were cultured in constant exponential growth as described previously \cite{Brooks1996}. 
Experimental cultures were initiated at $0.1\times10^6$ cells/mL 24 hr prior to 1$\mu$M ATRA treatment;
if indicated, cells were also treated with GW5074 (2$\mu$M) 18 hr before ATRA treatment.   
For cell culture washout experiments HL-60 cells were treated with ATRA for 24 hr, 
washed 3x with prewarmed serum supplemented culture medium to remove ATRA exposure, 
and reseeded in ATRA-free media as described. 
Western blot analysis was performed at incremental time points after removal of ATRA.

\subsection*{Chemicals.}
All-Trans Retinoic Acid from Sigma-Aldrich (St. Louis, MO) was dissolved in 100\% ethanol with a stock concentration of 5mM, 
and used at a final concentration of 1$\mu$M.
The Raf inhibitor GW5074 from Sigma-Aldrich (St. Louis, MO) was dissolved in DMSO with a stock concentration of 10mM, 
and used at a final concentration of 2$\mu$M.
HL-60 cells were treated with 2$\mu$M GW5074 with or without ATRA (1$\mu$M) at 0 hr.  
This GW5074 dosage had a negligible effect on the cell cycle distribution, compared to ATRA treatment alone (Fig. \ref{F:MAPKGWWB}A). 

\subsection*{CD11b expression studies by flow cytometry.}
Approximately $1.0\times10^6$ HL-60 cells were harvested  by centrifugation. 
Cells were resuspended in 200$\mu$L PBS containing 5$\mu$L of allophycocyanin (APC)–conjugated anti-CD11b antibody from BD Biosciences (San Jose, CA). 
Following incubation for 1hr at 37$^{o}$C, cells were analyzed by flow cytometry (LSRII flow cytometer, BD Biosciences; San Jose, CA) using 633nm red laser excitation. 
The threshold on experimental groups was set to exclude 95\% of control, the untreated sample. 

\subsection*{Measurement of inducible oxidative metabolism.}
Approximately $0.5\times10^6$ Cells were harvested by centrifugation.
Cells were resuspended in 200$\mu$L 37$^{o}$C PBS containing 10$\mu$mol/L 
5-(and-6)-chloromethyl-2,7-dichlorodihydro-fluorescein diacetate acetyl ester (DCF; Invitrogen Carlsbad, CA) 
and 0.4$\mu$g/mL 12-O-tetradecanoylphorbol-13-acetate (TPA; Sigma-Aldrich St. Louis, MO)
with incubation for 20 min in a humidified atmosphere of 5\% CO$_2$ at 37$^{o}$C. 
Flow cytometric analysis was done as described previously \cite{Reiterer2007}.
To determine TPA inducible ROS, the threshold on experimental groups was set to exclude 95\% of control,
samples not treated with TPA.  

\subsection*{Cell cycle analysis.}
Approximately $1.0\times10^6$ cells were collected by centrifugation.
Cells were resuspended in 500$\mu$L hypotonic staining solution containing 50$\mu$g/mL propidium iodine, 
1$\mu$L/mL Triton X-100, and 1 mg/mL sodium citrate. 
Cells were incubated at room temperature for 1 hr and analyzed by flow cytometry (BD LSRII) using 488-nm excitation.

\subsection*{Immunoprecipitation and Western blot.}
Approximately $1.2 \times 10^7$ cells were lysed using 400$\mu$L of M-Per lysis buffer from Thermo Scientific (Waltham, MA). 
Lysates were cleared by centrifugation at 16,950 $\times$ g in a micro-centrifuge for 20 min at 4$^{o}$C.  
Lysates were pre-cleared using 100$\mu$L protein A/G Plus agarose beads from Santa Cruz Biotechnology (Santa Cruz, CA) by 
inverting overnight at 4$^{o}$C.  Beads were cleared by centrifugation and total protein concentration
was determined by a BCA assay (Thermo Scientific, Waltham, MA).  Immunoprecipitations were setup by bringing 
lysate to a concentration of 1.0g/L in a total volume of 300$\mu$L (M-Per buffer was used for dilution).     
The anti-Raf antibody was added at 3 $\mu$L.
A negative control with no bait protein was also used to exclude the direct interaction of proteins with the A/G beads.   
After 1 hr of inversion at 4$^{o}$C, 20$\mu$L of agarose beads was added and samples were left 
to invert overnight at 4$^{o}$C.  
Samples were then washed three times with M-Per buffer by centrifugation.
Finally proteins were eluted from agarose beads using a laemmli loading buffer.
Eluted proteins were resolved by SDS-PAGE and Western blotting.  Total 
lysate samples were normalized by total protein concentration (20$\mu$g per sample)
and resolved by SDS-PAGE and Western blotting.  Secondary HRP bound antibody was used for visualization.
All antibodies were purchased from Cell Signaling (Boston, MA) with the exception of anti-p621 Raf which was purchased from Biosource/Invitrogen  (Carlsbad, CA), and
anti-pS338 Raf which was purchased from Santa Cruz Biotechnology (Santa Cruz, CA); anti-retinoblastoma from Zymed (South San Francisco, CA); and anti-CK2 from BD Biosciences (San Jose, CA).
     
\subsection*{Formulation and solution of the model equations.}
We used ordinary differential equations (ODEs) to model the time evolution of the circuit components:
\begin{equation}
	\frac{dx_{i}}{dt}  = \sum_{j = 1}^{\mathcal{R}}\sigma_{ij}r_{j}\left(\mathbf{x},\mathbf{\epsilon},\mathbf{k}\right) - \left(\mu + k_{d,i}\right)x_{i}\qquad{i=1,2,\hdots,\mathcal{M}}\\
\end{equation}where $\mathcal{R}$ denotes the number of reactions, $\mathcal{M}$ denotes the number of circuit components, $k_{d,i}$ denotes a first-order non-specific degradation constant
for species i, and $\mu$ denotes the specific growth rate of HL-60 cells.
The quantity $\sigma_{ij}$ denotes the stoichiometric coefficient for species $i$ in reaction $j$. If 
$\sigma_{ij}>0$, metabolite $i$ is produced by reaction $j$. Conversely, if $\sigma_{ij}>0$, metabolite $i$ is consumed by reaction $j$, while $\sigma_{ij} = 0$ indicates
metabolite $i$ is not connected with reaction $j$. 
The quantity $r_{j}\left(\mathbf{x},\mathbf{\epsilon},\mathbf{k}\right)$ denotes the rate of reaction $j$. 
Typically, reaction $j$ is a non-linear function of network state, as well as unknown kinetic parameters $\mathbf{k}$ ($\mathcal{K}\times{1}$).
The system material balances were subject to the initial conditions $\mathbf{x}\left(t_{o}\right) = \mathbf{x}_{o}$.

Each reaction rate was written as the product of two terms, a kinetic term ($\bar{r}_{j}$) and a regulatory term ($v_{j}$):
\begin{equation}\label{eqn:rate-factor}
	r_{j}\left(\mathbf{x},\mathbf{\epsilon},\mathbf{k}\right) = \bar{r}_{j}v_{j}
\end{equation}We used multiple saturation kinetics to model the reaction term $\bar{r}_{j}$:
\begin{equation}\label{eqn:rate-bar}
	\bar{r}_{j} = k_{j}^{max}\epsilon_{i}\left(\prod_{s\in{m_{j}^{-}}}\frac{x_{s}}{K_{js} + x_{s}}\right)
\end{equation}where $k_{j}^{max}$ denotes the maximum rate for reaction $j$, $\epsilon_{i}$ denotes the scaled enzyme activity which catalyzes reaction $j$, and
$K_{js}$ denotes the saturation constant for species $s$ in reaction $j$. 
The product in Eqn. \eqref{eqn:rate-bar} was carried out over the set of \textit{reactants} for reaction $j$ (denoted as $m_{j}^{-}$).

The control term $v_{j}$ depended upon the combination of factors which influenced the abundance or activity of $x_{i}$.
For each species, we used a rule based approach to select from competing control factors (Fig. \ref{fig-control-schematic}). 
We modeled species activation as:
\begin{equation}
	v_{j} = \max\left(f_{1j}\left(x\right),\hdots,f_{mj}\left(x\right)\right)
\end{equation}where $0\leq f_{ij}\left(x\right)\leq 1$ was a regulatory transfer function that calculated the influence of species $x_{i}$ on rate process $j$. 
Conversely, we modeled the inhibition (or repression) of rate process $j$ as:
\begin{equation}
	v_{j} = 1 - \max\left(f_{1j}\left(x\right),\hdots,f_{mj}\left(x\right)\right)
\end{equation}Lastly, if rate process $j$ was subject to had both $m$ activating and $n$ inhibitory factors, we modeled the control term as:
\begin{equation}
	v_{j} = \min\left(u_{j},d_{j}\right)
\end{equation}where:
\begin{eqnarray}
	u_{j} &=& \max_{j^{+}}\left(f_{1j}\left(x\right),\hdots,f_{mj}\left(x\right)\right) \\
	d_{j} &=& 1 - \max_{j^{-}}\left(f_{1j}\left(x\right),\hdots,f_{nj}\left(x\right)\right)
\end{eqnarray}The quantities $j^{+}$ and $j^{-}$ denoted the sets of activating, and inhibitory factors for process $j$. 
If a process had no control factors, we set $v_{j} = 1$.
There are many possible functional forms for $0\leq f_{ij}\left(x\right)\leq 1$. 
However, in this study, each individual transfer function took the form:
\begin{equation}\label{eqn:control-factor}
	f_{i}\left(\mathbf{x}\right) = \frac{\kappa_{ij}^{\eta}x_{j}^{\eta}}{1 + \kappa_{ij}^{\eta}x_{j}^{\eta}}
\end{equation}where $x_{j}$ denotes the abundance (or activity) of component $j$.  
The $\kappa_{ij}$ parameter was species gain parameter, while $\eta$ was a cooperativity parameter (similar to a Hill coefficient).
The model equations were encoded using the Octave programming language, and solved using the LSODE routine in Octave \citep{Octave:2014}.

\subsection*{Estimation of an ensemble of model parameters from experimental data.}
Model parameters were estimated by minimizing the squared difference between simulations and experimental data taken from ATRA-induced HL-60 cells:
\begin{equation}\label{eqn:objective-function}
	E_{j}(\mathbf{k}) = \sum_{i=1}^{\mathcal{T}_{j}}\biggl(\hat{\mathcal{M}}_{ij}-\hat{y}_{ij}(\mathbf{k})\biggr)^2 + \left(\frac{{\mathcal{M}^{\prime}_{ij}}-\max{y_{ij}}}{{\mathcal{M}^{\prime}_{ij}}}\right)^{2}
\end{equation}
The terms $\hat{\mathcal{M}}_{ij}$ and $\hat{y}_{ij}$ denote scaled experimental observations and simulation outputs from training set $j$. 
The quantity $i$ denoted the sampled time-index and $\mathcal{T}_{j}$ denoted the number of time points for experiment j. 
The first term in Eqn. \eqref{eqn:objective-function} quantified the relative error in the simulation. 
We used only immunoblot measurements for model training. 
Thus, we trained the model on the \emph{relative} change between bands within each training data set.
The read-out from the training immunoblots was band intensity where we assumed intensity was only loosely proportional to concentration.
Suppose we have the intensity for species $x$ at time $\{t_{1},t_{2},..,t_{n}\}$ in condition $j$. The scaled-value $\hat{\mathcal{M}}_{ij}$ would then be given by:
\begin{equation}\label{norm_exp_data}
\hat{\mathcal{M}}_{ij} = \frac{\mathcal{M}_{ij} - \min_{i}\mathcal{M}_{ij}}{\max_{i}{\mathcal{M}_{ij}}-\min_{i}{\mathcal{M}_{ij}}}
\end{equation}
Under this scaling $0\leq\hat{\mathcal{M}}_{ij}\leq{1}$ where $\hat{\mathcal{M}}_{ij}=0$ describes the lowest intensity band and $\hat{\mathcal{M}}_{ij}=1$ describes the highest intensity band. 
A similar scaling was defined for the simulation output. The second-term in the objective function ensured the proper concentration scale was estimated by the model.
In this study, we set the highest intensity band to $\mathcal{M}^{\prime}_{ij} = 10$ [AU] for all simulations. 

We minimized the total model residual ($\sum{E_{j}}$) using Particle swarm optimization (PSO) \citep{PSO}.
PSO uses a \textit{swarming} metaheuristic to explore parameter spaces. 
A strength of PSO is its ability to find the global minimum, even in the presence of potentially many local minima, by communicating the local
error landscape experienced by each particle collectively to the swarm. Thus, PSO acts both as a local and a global search algorithm. 
For each iteration, particles in the swarm compute their local error by evaluating the model equations using their specific parameter vector realization.
From each of these local points, a globally best error is identified. Both the local and global error 
are then used to update the parameter estimates of each particle using the rules:
\begin{eqnarray}
	\mathbf{\Delta}_{i} &=&\theta_{1}\mathbf{\Delta}_{i} + \theta_{2}\mathbf{r}_{1}\left(\mathcal{L}_{i} - \mathbf{k}_{i}\right) + \theta_{3}\mathbf{r}_{2}\left(\mathcal{G} - \mathbf{k}_{i}\right) \\
	\mathbf{k}_{i} &=& \mathbf{k}_{i} + \mathbf{\Delta}_{i}
\end{eqnarray}where $\left(\theta_{1},\theta_{2},\theta_{3}\right)$ are adjustable parameters, $\mathcal{L}_{i}$ denotes local best solution found by particle $i$, and
$\mathcal{G}$ denotes the best solution found over the entire population of particles. The quantities $r_{1}$ and $r_{2}$ denote uniform random vectors with the same dimension as the number of unknown model
parameters ($\mathcal{K}\times{1}$). In thus study, we used $\left(\theta_{1},\theta_{2},\theta_{3}\right) = \left(1.0, 0.05564, 0.02886\right)$, which was taken from XXX. 

\section*{Acknowledgements}
The authors thank the anonymous reviewers for their helpful suggestions. 
We acknowledge the gracious financial support to J.V. by the National Science Foundation CAREER (CBET-0846876) for the support of R.T. and H.J.
In addition, we acknowledge support to A.Y. from the National Institutes of Health (CA 30555, CA152870) 
and a grant from New York State Stem Cell Science.  Lastly, we acknowledge the financial support to J.V. and A.Y. 
from the National Cancer Institute (\#U54CA143876). The content is solely the responsibility of the authors and does not necessarily 
represent the official views of the National Cancer Institute or the National Institutes of Health.

\clearpage
%\bibliographystyle{plain}
%\bibliographystyle{IEEEbib}

\bibliography{References_v1}

\clearpage

% Supplemental figures -
% Set the S- 
\renewcommand\thefigure{S\arabic{figure}}
\renewcommand\thetable{T\arabic{table}}
\renewcommand\thepage{S-\arabic{page}}
\renewcommand\theequation{S\arabic{equation}}

% Reset the counters -
\setcounter{equation}{0}
\setcounter{table}{0}
\setcounter{figure}{0}
\setcounter{page}{1}

\end{document}

